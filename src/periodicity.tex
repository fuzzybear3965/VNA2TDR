\section{Periodicity Considerations}
\label{sec:periodicity_considerations}

\begin{framed}
   Performing a IDFT on the measured VNA data should use a periodic time-domain
   input because the DFT is a periodic transform.
\end{framed}

As acknowledged in Phillip Dunsmore's thesis
(\url{http://etheses.whiterose.ac.uk/3355/1/uk_bl_ethos_406207.pdf}) a periodic
transform imposes periodic functions on the analysis. The discrete Fourier
transform is such a periodic transform. Thus, both the time-domain response and
the frequency-domain response must be considered periodic. Quoting the thesis:
``The step response of the VNA should retain the property that its derivative is
the VNA time-domain impulse response, and since the sampling function [the VNA
being the sampler] creates a periodic time-domain, with a period of $
\frac{1}{\Delta \omega} $, the step response should retain this aspect of the
periodicity.'' From here, the author determines the requirements of the step
response given the required periodicity of the impulse response. And herein lies
an important detail: The step response is the sum of two responses, one is a
periodic portion (responsible for the periodicity of the impulse response); the
other part is a ramp portion (staircase), which is responsible for the impulses
within the impulse response. However, since this is not in $ L^2 $, a Laplace
transform is used for the ramp portion and a Fourier transform is used for the
periodic portion.

I think the above detail may be incredibly relevant.
